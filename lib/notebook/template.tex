\documentclass[a4paper,oneside,twocolumn]{article}
\usepackage[utf8]{inputenc}
\usepackage[english]{babel}

\usepackage[T1]{fontenc}
\linespread{1.185}

%\usepackage{eulervm}
\usepackage[landscape,left=.5cm,right=.5cm,top=2.1cm,bottom=.5cm]{geometry}

\setlength{\columnseprule}{0.25pt}
\usepackage{fancyhdr}
\pagestyle{fancy}
\fancyhf{}
\fancyhead[RE,RO]{ACM ICPC Reference, page \bfseries\thepage}
\fancyhead[LE,LO]{University of São Paulo}

\usepackage{listings}
%\usepackage{pxfonts}
\usepackage[scaled=1.14]{couriers}

\usepackage{sectsty}
\sectionfont{\normalsize}

\newcommand{\SECTION}[2]{\section*{#1} \addcontentsline{toc}{subsection}{#1} #2 \begin{center}\rule{400pt}{0.25pt}\end{center}}
\newcommand{\FINALSECTION}[1] {\section*{#1} \addcontentsline{toc}{subsection}{#1}}
\newcommand{\sourcefile}[1]{\begin{center}\textbf{#1}\end{center}\lstinputlisting{src/#1}}
%\newcommand{\Csourcefile}[1]{\begin{center}\textbf{#1.cpp}\end{center}\lstinputlisting[language=C, basicstyle=\ttfamily, keywordstyle=\bfseries, showstringspaces=false, morekeywords={ll}]{src/#1.cpp}}
\newcommand{\Csourcefile}[1]{\lstinputlisting[language=C, basicstyle=\ttfamily, keywordstyle=\bfseries, showstringspaces=false, morekeywords={ll}]{src/#1.cpp}}

\usepackage{amsmath}
\usepackage{amsfonts}
\usepackage{amssymb}
\usepackage{amsthm}
\newcommand{\NN}{\mathbb{N}}
\newcommand{\ZZ}{\mathbb{Z}}
\newcommand{\QQ}{\mathbb{Q}}
\newcommand{\RR}{\mathbb{R}}
\newcommand{\then}{\Longrightarrow}
\newcommand{\floor}[1]{\left\lfloor#1\right\rfloor}
\newcommand{\ceil}[1]{\left\lceil#1\right\rceil}
\newcommand{\paren}[1]{\left(#1\right)}
\newcommand{\brackets}[1]{\left[#1\right]}
\newcommand{\braces}[1]{\left\{#1\right\}}
\newcommand{\abs}[1]{\left\lvert#1\right\rvert}
\newcommand{\nequiv}{\not\equiv}
\newcommand{\ds}{\displaystyle}
\newcommand{\bigO}{\mathcal{O}}
\newcommand{\norm}[1]{\abs{\abs{#1}}}
\newcommand{\degree}{\ensuremath{^\circ}}
\newcommand{\defun}[5] {
    \begin{array}{rrcl}
#1: & #2 & \longrightarrow & #3 \\
    & #4 & \longmapsto & #5
    \end{array}
}
\renewcommand{\le}{\leqslant}
\renewcommand{\ge}{\geqslant}
\renewcommand{\epsilon}{\varepsilon}

\newcommand{\stirfst}[2]{\genfrac{[}{]}{0pt}{}{#1}{#2}}
\newcommand{\stirsnd}[2]{\genfrac{\{}{\}}{0pt}{}{#1}{#2}}
\newcommand{\bell}[1]{\mathcal B_{#1}}
\newcommand{\seg}[1]{\overline{#1}}

\newcommand{\bmat}[1]{\begin{bmatrix}#1\end{bmatrix}}
\newcommand{\vmat}[1]{\begin{vmatrix}#1\end{vmatrix}}

\title{ACM ICPC Reference}
\author{University of São Paulo}

%%%%%%%%%%%%%%%%%%%%%%%%%%%%%%%%%%%%%%%%%%%%%%%%%%%%%%%%%%%%%
\begin{document}
\scriptsize{}
\maketitle
\thispagestyle{fancy}


\tableofcontents

%%%%%%%%%%%%%%%%%%%%%%%%%%%%%%%%%%%%%%%%%%%%%%%%%%%%%%%%%%%%%%%%%%%%%%%%%%%%%%%%%%%%%%
%%%%%%%%%%%%%%%%%%%%%%%%%%%%%%%%%%%%%%%%%%%%%%%%%%%%%%%%%%%%%%%%%%%%%%%%%%%%%%%%%%%%%%

\newpage

\section{Template} {
  \sourcefile{iniciodeprova.txt}
  \Csourcefile{hashed/template}
  \sourcefile{../hashify.py}
}
\section{Lista de bugs e recomendações} {
  \begin{itemize} \itemsep1pt \parskip0pt \parsep0pt
  \item Reler enunciado e pedir clarifications (evitar explicar o enunciado para outra pessoa).
  \item Verificar overflows.
  \item Ver se o $\infty$ é tão infinito quanto parece e se o $eps$ é tão pequeno quanto o necessário.
  \item Comparação de ponto flutuante com tolerância.
  \item Verificar se o grafo pode ser desconexo.
  \item Verificar se pode haver self-loops, arestas com peso negativo ou ligando um mesmo par de vértices.
  \item Cuidar de casos com pontos coincidentes e pontos colineares.
  \item Igualdade dentro de if (a == b ao invés de a = b)
  \item Verificar trechos de código quase iguais ou copy-pasted.
  \item Verificar tamanho de vetores.
  \item Overflow em shift (1ll $<<$ 40 ao invés de 1 $<<$ 40)
  \item Não usar variáveis com nome min, max, next.
  \item Verificar inicialização de variáveis.
  \item Verificar casos extremos, muito pequenos ou muito grandes, caso zero.
  \item Cuidar de imprecisões ao subtrair números quase iguais.
  \item Tomar cuidado com resto de divisão envolvendo números negativos.
  \item Não comparar unsigned int (.size()) com int negativo.
  \end{itemize}
}
  
\section{Numbers and Number Theory}

\SECTION{Extended Euclid's Algorithm (Bézout's Theorem)}{
    \Csourcefile{hashed/bezout}
}

\SECTION{Chinese Remainder Theorem } {
Suppose $m_0, m_1, ..., m_{k-1}$ are positive integers that are pairwise coprime. Then, for any given sequence of integers $a_1, a_2, ..., a_{k-1}$, there exists an integer $X$ solving the following system of simultaneous congruences.
	\[ X \equiv a_0 \pmod{m_0} \]
    \[ X \equiv a_1 \pmod{m_1} \]
    \[ \vdots \]
   	\[ X \equiv a_{k-1} \pmod{m_{k-1}} \] 
Furthermore, all solutions $X$ of this system are congruent modulo the product, $M = m_0m_1...m_{k-1}$. 
	A possible solution for this system is: 
	
	$X = \sum_{i=0}^{k-1}\frac{M}{m_i}b_ia_i $, where $b_i$ is a integer, such that: 
	$\forall i, 0 \leq i \leq k-1, \frac{M}{m_i}b_i \equiv 1 \pmod{m_i} $.
	
	$\linebreak$	
	OBS.: Observações sobre chinesResto.cpp.	
	\begin{itemize} \itemsep1pt \parskip0pt \parsep0pt
	\item Não usa o algoritmo mostrado acima;
	\item Se o sistem tem solução, armazena em X o valor da menor solucao positiva e retorna true. Retorna false caso contrário;
	\item Complexidade : O(k * log( max{m[i]} ), sem usar mulmod;
	\item CUIDADO: l = lcm(m[0],m[1],...,m[k-1]) cuidado para não estourar long long;
	\item CUIDADO: os valores do vetor a[0...k-1] são alterados;
	\item CUIDADO: assume que k > 1;
	
	\end{itemize}
	\Csourcefile{hashed/chinesResto}
}


%% \SECTION{62-bit Modular Multiplication / Exponentiation}{
%%     \Csourcefile{hashed/mulmod}
%%     \Csourcefile{hashed/expmod}
%% }

\SECTION{Miller--Rabin (Primality test)}{
    \Csourcefile{hashed/isprime}
}

\SECTION{Pollard's Rho (Factorization)}{
   \Csourcefile{hashed/rho}
}


\SECTION{Brent's Algorithm (Cycle detection)}{
    Let $x_0\in S$ be an element of the finite set $S$ and consider a function $f:S\to S$. Define
        \[ f_k(x) = \begin{cases}x,& k = 0 \\ f\bigl(f_{k-1}(x)\bigr),& k > 0\end{cases}. \]

        Clearly, there exists distinct numbers $i,j\in\NN$, $i\neq j$, such that $f_i(x_0) = f_j(x_0)$.

        Let $\mu\in\NN$ be the least value such that there exists $j\in\NN\setminus\{\mu\}$ such that
        $f_\mu(x_0) = f_j(x_0)$ and let $\lambda\in\NN$ be the least value such that
        $f_\mu(x_0) = f_{\mu+\lambda}(x_0)$.

        Given $x_0$ and $f$, this code computes $\mu$ and $\lambda$ applying the operator $f$
        $\bigO(\mu+\lambda)$ times and storing at most a constant amount of elements from $S$.

        %\Csourcefile{brent}
}

\SECTION{Lucca's Theorem} {
	For non-negative integers $m$ and $n$ and a prime $p$, the following congruence relation holds:
	
	\[ \binom mn \equiv \prod_{i=0}^k \binom{m_i}{n_i} \pmod p, \]
	
	where
	
	$ m = m_kp^k + m_{k-1}p^{k-1} + ... + m_1p + m_0$
	
	and
	
	$ n = n_kp^k + n_{k-1}p^{k-1} + ... + n_1p + n_0$		
	
	are the base $p$ expansions of $m$ and $n$ respectively.
}

\SECTION {Simpson's Rule for numerical integration} {
Simpson's rule is a method for numerical integration, the numerical approximation of definite integrals. 
	\[  \int_a^b f(x)\,\mathrm{d}x \approx \frac{b-a}6 \Bigg[ f(a) + 4f(\frac{a+b}2) + f(b) \Bigg] \]

	\begin{itemize} \itemsep1pt \parskip0pt \parsep0pt
	\item Suppose that the interval $[a, b]$ is split up in $n$ subintervals, with $n$ an even number. Then, the composite Simpson's rule is given by:
	
	\[ \int_a^b f(x)\,\mathrm{d}x \approx \frac{h}3 \Bigg[ f(x_0) + 2\sum_{j=1}^{\frac{n}2-1} f(x_{2j}) + 4\sum_{j=1}^{\frac{n}2} f(x_{2j-1}) + f(x_n) \Bigg]   \]
	Where $x_j=a+jh$ for $j=0, 1, ..., n-1, n$,  with $h=(b-a)/n$ (in particular, $x_0=a$ and $x_n=b$). 
	
	The above formula can also be written as:
	\[ \int_a^b f(x)\,\mathrm{d}x \approx \frac{h}3 \Bigg[f(x_0) + 4f(x_1) + 2f(x_2) + 4f(x_3) + 2f(x_4) + ... + 4f(x_{n-1}) + f(x_n) \Bigg]   \]
	
	%\sum_{i=1}^{10} t_i
	\end{itemize}
}

\SECTION {Lagrange Interpolating Polynomial} {
  The Lagrange interpolating polynomial is the polynomial $P(x)$ of degree $\leq (n-1)$ that passes through the $n$ points $(x_1,y_1=f(x_1)),  (x_2,y_2=f(x_2)), ..., (x_n,y_n=f(x_n))$, and is given by
  \[ P(x) = \sum_{i=1}^{n}y_iP_i(x), \]
  where
  \[ P_i(x) = \prod_{\substack{k=1 \\ k\neq j}}^{n}\frac{x-x_k}{x_i-x_k} \]
  Written explicitly,
  \[ \begin{split} P(x) = \frac{(x - x_2)(x - x_3)...(x - x_n)}{(x_1 - x_2)(x_1 - x_3)...(x_1 - x_n)}y_1 &+
    \frac{(x - x_1)(x - x_3)...(x - x_n)}{(x_2 - x_1)(x_2 - x_3)...(x_2 - x_n)}y_2 + ... \\
    &+ \frac{(x - x_1)(x - x_2)...(x - x_{n-1})}{(x_n - x_1)(x_n - x_2)...(x_n - x_{n-1})}y_n. \end{split}\]
}

\SECTION {Farey Sequence} {
 Farey Sequence of order $n$ is the sequence of completely reduced fractions between $0$ and $1$ which, when in lowest terms, have denominators less than or equal to $n$, arranged in order of increasing size.
 
    $ F_1 =\{ {0}/{1}, {1}/{1}\}, \enskip F_2=\{ 0/1, 1/2, 1/1\}, \enskip F_3=\{ 0/1, 1/3, 1/2, 2/3, 1/1\},...  $
 
 	\begin{itemize} \itemsep1pt \parskip0pt \parsep0pt
	\item From this, we can relate the lengths of $F_n$ and $F_{n-1}$ using Euler's totient function $\varphi(n)$ :  
		\[  |F_n| = |F_{n-1}| + \varphi(n) \]

	\item The asymptotic behaviour of $|F_n|$ is : $|F_n|  \sim \frac{3n^2}{\pi^2}$.
	\end{itemize}
	 
	\Csourcefile{hashed/farey}
}

 
\SECTION {Euler's Totient Function} {
	\[ \varphi(n) = n\prod_{p|n} \Bigg(1 - \frac{1}p \Bigg)  ; \quad\quad
	   \varphi(p.n) = \begin{cases}(p-1)\varphi(n),& p\dagger n \\ p\varphi(n),& p \mid n \end{cases}  ;\]
	\[ \varphi(mn) = \varphi(m)\varphi(n).\frac{d}{\varphi(d)} , where\enskip d = gcd(m, n).\enskip Note\enskip the\enskip special\enskip cases. \]
	
	\[ \varphi(p^k) = p^k\Bigg(1 - \frac{1}p \Bigg) ; \quad\quad
	   \sum_{d|n} \varphi(d) = n ;\]
	Euler's theorem : if $a$ and $n$ are relatively prime then
	\[a^{\varphi(n)} \equiv 1 \pmod n \]
}


\SECTION {Pillai's arithmetical function} {
	\[ P(n) := \sum_{k=1}^n gcd(k, n) \enskip (n \in \mathbb{N} := \{1,2,...\})\]
	\[ P(n) = \sum_{d|n}d\varphi(n/d) \enskip (n \in \mathbb{N}) \]
	
	For every prime power $p^a$, $(a\in \mathbb{N})$, $P(p^a) = (a + 1)p^a - ap^{a-1}$.
	
	$P(n)$ is multiplicative, ie, for every integers $a$ and $b$, such that $gcd(a,b) = 1$, $P(ab) = P(a)P(b)$.

	Let $P_{altern}(n)$ be the alternating gcd-sum function. Let $n\in \mathbb{N}$ and write $n = 2^am$, where $a\in \mathbb{N}_0 := 
	\{0,1,2,....\}$ and $m\in \mathbb{N}$ is odd. Then
	
	\[ P_{altern}(n) := \sum_{k-1}^{n} (-1)^{k-1}gcd(k,n) = 
	\begin{cases}n,& $if n is odd; $ \\ -2^{a-1}aP(m) =  -\frac{a}{a+2}P(n) ,& $if n is even.$ \end{cases}  \]
}


\SECTION {Gaussian Elimination} {
  Can be easily adapted to integers mod P, fractions and long doubles.
  \Csourcefile{hashed/gaussian}
}

\SECTION {Fast Fourier Transform} {
  If you are calculating the product of polynomials, don't forget to set the vector's size to \textbf{at least the sum of degrees} of both polynomials, regardless of whether you will use only the first few elements of the array.
  \Csourcefile{hashed/fft}
}

\SECTION {Least Squares Method} {
  
  Let $f, g_0, g_1, ..., g_{n-1}$ be given functions. The coefficients of the function $g(x) = a_0 g_0(x) + a_1 g_1(x) + ... + a_{n-1} g_{n-1}(x)$ that best fits $f(x)$ so that the inner product $<f-g, f-g>$ is minimum is given by the system of equations below.

  \[ \left[ \begin{array}{cccc}
      <g_0, g_0> & <g_0, g_1> & ... & <g_0, g_{n-1}> \\
      <g_1, g_0> & <g_1, g_1> & ... & <g_1, g_{n-1}> \\
      \vdots & \vdots & \ddots & \vdots \\
      <g_{n-1},g_0> & <g_{n-1}, g_1> & ... & <g_{n-1}, g_{n-1}> \end{array} \right]
  \left[ \begin{array}{c}
      a_0 \\
      a_1 \\
      \vdots \\
      a_{n-1} \end{array} \right] =
  \left[ \begin{array}{c}
      <f, g_0> \\
      <f, g_1> \\
      \vdots \\
      <f, g_{n-1}> \end{array} \right] \]

  In the linear discrete case when $g(x) = ax + b$, $a$ and $b$ can also be found using the following equations:

  \[ a = \frac{\sum_{i=1}^n x_i(y_i - \bar{y})}{\sum_{i=1}^n x_i(x_i - \bar{x})}, \hspace{1cm} b = \bar{y} - a\bar{x} \]
  where $\bar{x}$ and $\bar{y}$ are the mean value of $x$ and $y$, respectively.
}
  

%%%%%%%%%%%%%%%%%%%%%%%%%%%%%%%%%%%%%%%%%%%%%%%%%%%%%%%%%%%%%%%%%%%%%%%%%%%%%%%%%%%%%%
%%%%%%%%%%%%%%%%%%%%%%%%%%%%%%%%%%%%%%%%%%%%%%%%%%%%%%%%%%%%%%%%%%%%%%%%%%%%%%%%%%%%%%
\section{Combinatorics}

\SECTION{Catalan Numbers} {
    $C_n$ is:
        \begin{itemize} \itemsep1pt \parskip0pt \parsep0pt
    \item The number of balanced expressions built from $n$ pairs of parentheses.
        \item The number of paths in an $n\times n$ grid that stays on or below the diagonal.
        \item The number of words of size $2n$ over the alphabet $\Sigma = \{a,b\}$ having an equal number of
        $a$ symbols and $b$ symbols containing no prefix with more $a$ symbols than $b$ symbols.
       	\item (starting with $C_0$): 1, 1, 2, 5, 14, 42, 132, 429, 1430, 4862, 16796, 58786, 208012, 742900, 2674440, 9694845, 35357670, 129644790, 477638700, 1767263190, 6564120420, 24466267020, 91482563640, 343059613650, 1289904147324, 4861946401452...
        \end{itemize}

    It holds that:
        \[ C_0 = 1, C_{n+1} = \sum_{k=0}^n C_k C_{n-k} \]
        \[ C_n = \binom{2n}n - \binom{2n}{n-1} = \frac 1{n+1}\binom{2n}n = \frac{(2n)!}{n!(n+1)!} \]

}

\SECTION{Stirling Numbers of the First Kind} {
    $\ds \stirfst nk$ is:
        \begin{itemize} \itemsep1pt \parskip0pt \parsep0pt
   		\item For integers $n \geq k \geq 0$, $\ds \stirfst nk$ counts the number of permutations of n elements with exactly k cycles.
        \item The number of digraphs with $n$ vertices and $k$ cycles such that each vertex
        has in and out degree of 1.
        \end{itemize}

    It holds that:
        \[ \stirfst n0 = \begin{cases}1,& n=0 \\ 0,& n\neq 0\end{cases}, \quad
           \stirfst 0k = \begin{cases}1,& k=0 \\ 0,& k\neq 0\end{cases} ; \quad\quad
           \stirfst nk = (n-1)\stirfst{n-1}k + \stirfst{n-1}{k-1} ; \]
        \[ \stirfst n1 = (n-1)! ; \quad\quad
           \stirfst n{n-1} = \binom n2 ; \quad\quad
           \stirfst n{n-2} = \frac 14 (3n-1) \binom n3 ; \]
        \[ \stirfst n{n-3} = \binom n2 \binom n4 ; \quad\quad
           \stirfst n2 = (n-1)! H_{n-1} ; \quad\quad
           \stirfst n3 = \frac 12 (n-1)! \left( H_{n-1}^2 - H_{n-1}^{(2)} \right) ; \]
        \[ H_n = \sum_{j=1}^n \frac 1j,\quad H_n^{(k)} = \sum_{j=1}^n \frac 1{j^k} ; \quad\quad
           \sum_{k=0}^n \stirfst nk = n! ; \quad\quad
           \sum_{j=k}^n \stirfst nj \binom jk = \stirfst{n+1}{k+1} ;\]
}

\SECTION{Stirling Numbers of the Second Kind} {
    $\ds \stirsnd nk$ is the number of ways to partition an $n$-set into exactly $k$ non-empty disjoint subsets
        up to a permutation of the sets among themselves. It holds that:

        \[ \stirsnd n0 = \begin{cases}1,& n=0 \\ 0,& n\neq 0\end{cases},\quad
        \stirsnd n1 = \stirsnd nn = 1 \]
        \[ \stirsnd nk = \stirsnd{n-1}{k-1} + k\stirsnd{n-1}k \]
        \[ \stirsnd nk \bmod 2 = \begin{cases}
    0,& (n-k)\&\floor{\frac{k-1}2}\neq 0 \\
        1,& \text{otherwise}
    \end{cases}, \]
        where $\&$ is the C bitwise ``and'' operator.

        \[ \stirsnd n2 = 2^{n-1} - 1 ; \quad\quad
           \stirsnd n{n-1} = \binom n2 ; \quad\quad
           \stirsnd nk = \frac 1{k!} \sum_{j=0}^k(-1)^{k-j} \binom kj j^n ;\]
}

\SECTION{Bell Numbers} {
    $\bell n$ is the number of equivalence relations on an $n$-set or, alternatively, the number of
        partitions of an $n$-set. It holds that:

        \[ \bell n = \sum_{k=0}^n \stirsnd nk ; \quad\quad
           \bell{n+1} = \sum_{k=0}^n\binom nk \bell k ; \quad\quad
           \bell n = \frac 1e \sum_{k=0}^\infty \frac{k^n}{k!} ;\]
        \[ \bell{n+p} \equiv \bell n + \bell{n+1} \pmod p \]
       \begin{itemize}  
       \item (starting with $\bell 0$): 1, 1, 2, 5, 15, 52, 203, 877, 4140, 21147, 115975, 678570, 4213597, 27644437, 190899322, 1382958545, 10480142147, 82864869804, 682076806159, 5832742205057, 51724158235372, 474869816156751, 4506715738447323, 44152005855084346, 445958869294805289, 4638590332229999353, ...
		\end {itemize}
}

\SECTION{Narayana Numbers} {
 	$N(n, k)$, $n = 1, 2, 3 ..., 1 \leq k \leq n$, is the number of expressions containing n pairs of parentheses which are correctly matched and which contain k distinct nestings. For instance, N(4, 2) = 6 as with four pairs of parentheses six sequences can be created which each contain two times the sub-pattern '()': ()((())) \enskip (())(()) \enskip (()(())) \enskip ((()())) \enskip ((())()) \enskip ((()))().
 		\[ N(n, k) = 0, \enskip if \enskip k > n \]
 		\[ N(n, 1) = N(n, n) = 1 \]
        \[ N(n, k) = \frac{1}{n}\binom nk \binom n{k-1} \]
       % \[ \binom nk \equiv \prod_{j=0}^\infty \binom{n_j}{k_j} \pmod p, \]
       \[ N(n, k) = \frac{n(n-1)}{(n-k)(n-k+1)} N(n-1, k) \]
       \[ N(n, 1) + N(n, 2) + N(n, 3) + ... + N(n, n) = C_n (Catalan\enskip Number) \]
}


\SECTION{The Twelvefold Way} {
    Let $A$ be a set of $m$ balls and $B$ be a set of $n$ boxes. The following table provides methods
        to compute the number of equivalent functions $f:A\to B$ satisfying specific constraints.

        \begin{tabular}{|c|c||c|c|c|}
    \hline
        Balls & Boxes & Any & Injective & Surjective \\
        \hline\hline
        $\nequiv$ & $\nequiv$ & $\ds n^m$ & $\ds\frac{n!}{(n-m)!}$ & $\ds n!\stirsnd mn$ \\
        \hline
        $\nequiv$ & $\equiv$ & $\ds \sum_{k=0}^n \stirsnd mk$ & $\ds\delta_{m\leqslant n}$ &
        $\ds\stirsnd mn$ \\
        \hline
        $\equiv$ & $\nequiv$ & $\ds\binom{m+n-1}{m}$ & $\ds\binom nm$ & $\ds\binom{m-1}{n-1}$ \\
        \hline
        $\equiv$ & $\equiv$ & {\bf(*)}$\ds\sum_{k=0}^n p(m,k)$ & $\ds\delta_{m\leqslant n}$ & 
        {\bf(**)}$\ds p(m,n)$ \\
    \hline
        \end{tabular}

    {\bf(**)} is a definition and both {\bf(*)} and {\bf(**)} are very hard to compute. So do not
    try to.
}


\SECTION{Partition (number theory)} {
	A partition of a positive integer $n$, is a way of writing $n$ as a sum of positive integers. Two sums that differ only in the order of their summands are considered to be the same partition.
	The number of partitions of n is given by the partition function $p(n)$. 
	Ex: $p(4) = 5$  => \{1+1+1+1\} , \{1+1+2\} , \{1+3\} , \{2+2\} , \{4\}.
	
	
    \begin{itemize} \itemsep1pt \parskip0pt \parsep0pt
    \item The number of partitions of $n$ in which the greatest part is $m$ is equal to the number of partitions of $n$ into $m$ parts.
    \item The number of partitions of n in which all parts are 1 or 2 (or, equivalently, the number of partitions of n into 1 or 2 parts) is $\lfloor \frac{n}{2} + 1 \rfloor$.
   	\item (starting with $p(0) = 1$): 1, 1, 2, 3, 5, 7, 11, 15, 22, 30, 42, 56, 77, 101, 135, 176, 231, 297, 385, 490, 627...
   	\item Summation to calculate $p(n)$ for the first $n$ elements in $O( n\sqrt{n})$ time.
    $p(n) = \sum_{k}(-1)^{k-1}p(n-\frac{k(3k-1)}{2})$, where the summation is over all \underline {\textbf {nonzero}} integers k (positive and negative) and p(m) is taken to be 0 if $m < 0$.
    \end{itemize}

	$p(5k + 4) \equiv 0 \pmod 5$
	
	$p(7k + 5) \equiv 0 \pmod 7$
	
	$p(11k + 6) \equiv 0 \pmod {11} $
		
	$p(11^3.13.k + 237) \equiv 0 \pmod {13} $	
	
}


\SECTION{Derangement (Desarranjo)} {
  A derangement is a permutation of the elements of a set such that none of the elements appear in their original position.

        Suppose that there are $n$ persons numbered $1, 2, \hdots, n$. Let there be $n$ hats also numbered $1, 2, \hdots, n$. 
        We have to find the number of ways in which no one gets the hat having same number as his/her number. 
        Let us assume that first person takes the hat $i$. There are $n - 1$ ways for the first person to choose the number $i$. Now there are 2 options:

        \begin{itemize} \itemsep1pt \parskip0pt \parsep0pt
    \item Person $i$ takes the hat of 1. Now the problem reduces to $n - 2$ persons and $n - 2$ hats.
        \item Person $i$ does not take the hat 1. This case is equivalent to solving the problem with $n - 1$ persons $n - 1$ hats (each of the remaining $n - 1$ people has precisely 1 forbidden choice from among the remaining $n - 1$ hats).
        \end{itemize}

    From this, the following relation is derived:

        $$d_n = (n-1) * (d_{n-1} + d_{n-2})$$
        $$d_1 = 0$$
        $$d_2 = 1$$

        Starting with $n = 0$, the numbers of derangements of $n$ are:
        1, 0, 1, 2, 9, 44, 265, 1854, 14833, 133496, 1334961, 14684570, 176214841, 2290792932.
}
%%%%%%%%%%%%%%%%%%%%%%%%%%%%%%%%%%%%%%%%%%%%%%%%%%%%%%%%%%%%%%%%%%%%%%%%%%%%%%%%%%%%%%
%%%%%%%%%%%%%%%%%%%%%%%%%%%%%%%%%%%%%%%%%%%%%%%%%%%%%%%%%%%%%%%%%%%%%%%%%%%%%%%%%%%%%%

\vspace{1cm}

\section {Geometry 2D} 
\SECTION {Point Structure}{
	\Csourcefile{hashed/point}
}

\SECTION {Auxiliary Functions} {
	\Csourcefile{hashed/geometriaFuncoes}
}
\SECTION {Convex Hull} {
	\Csourcefile{hashed/convexHullDouble}
}

\vspace{1cm}

\SECTION {Convex Polygon Intersection} {
	\Csourcefile{hashed/polyIntersect}
}

\SECTION {Polygon-Line Intersection} {
	\Csourcefile{hashed/polygonLine}
}

\SECTION {Polygon Triangulation} {
	\Csourcefile{hashed/polygonTriangulation}
}

\SECTION {Closest Pair} {
	\Csourcefile{hashed/closestPair}
}

\SECTION {Crosses Half-Plane} {
	\Csourcefile{hashed/crossHalfPlane}
}

\SECTION {N segments Intersection} {
	\Csourcefile{hashed/segmentIntersection}
}

\SECTION {Circle Structure} {
	\Csourcefile{hashed/circleStructure}
}

\SECTION {Circle Intersection} {
	\Csourcefile{hashed/circleIntersection}
}

\SECTION {Circle Union} {
    \Csourcefile{hashed/CircleUnionArea}
}

\SECTION {Minimum Spanning Circle} {
	\Csourcefile{hashed/spanningCircle}
}

\SECTION {Circle-Polygon Intersection} {
	\Csourcefile{hashed/circlePoly}
}


%%%%%%%%%%%%%%%%%%%%%%%%%%%%%%%%%%%%%%%%%%%%%%%%%%%%%%%%%%%%%%%%%%%%%%%%%%%%%%%%%%%%%%
%%%%%%%%%%%%%%%%%%%%%%%%%%%%%%%%%%%%%%%%%%%%%%%%%%%%%%%%%%%%%%%%%%%%%%%%%%%%%%%%%%%%%%
\section {Geometry 2D - Integer} 
\SECTION {Geometry Integer} {
	\Csourcefile{hashed/pointInteger}
}



%%%%%%%%%%%%%%%%%%%%%%%%%%%%%%%%%%%%%%%%%%%%%%%%%%%%%%%%%%%%%%%%%%%%%%%%%%%%%%%%%%%%%%
%%%%%%%%%%%%%%%%%%%%%%%%%%%%%%%%%%%%%%%%%%%%%%%%%%%%%%%%%%%%%%%%%%%%%%%%%%%%%%%%%%%%%%
\section {Geometry 3D} 
\SECTION {Geometry 3D} {
	\Csourcefile{hashed/point3D}
}



%%%%%%%%%%%%%%%%%%%%%%%%%%%%%%%%%%%%%%%%%%%%%%%%%%%%%%%%%%%%%%%%%%%%%%%%%%%%%%%%%%%%%%
%%%%%%%%%%%%%%%%%%%%%%%%%%%%%%%%%%%%%%%%%%%%%%%%%%%%%%%%%%%%%%%%%%%%%%%%%%%%%%%%%%%%%%

\section{Graph}

\SECTION {Erdős–Gallai theorem} {
A sequence of non-negative integers $d_1\geq\cdots\geq d_n$ can be represented as the degree sequence of a finite simple graph on $n$ vertices if and only if $d_1+\cdots+d_n$ is even and

   \[ \sum^{k}_{i=1}d_i\leq k(k-1)+ \sum^n_{i=k+1} \min(d_i,k) \]

holds for $1\leq k\leq n$.
}

\SECTION {Stable Marriage} {
  \Csourcefile{hashed/stable}
}

\vspace{1cm}

\SECTION {Dinic maxflow} {
  \Csourcefile{hashed/dinic}
}

\SECTION {Min-cost maxflow} {
  \Csourcefile{hashed/minCostMaxFlow}
}

\SECTION {Finding one solution for 2SAT} {
%%  \Csourcefile{hashed/sol2SAT}
  \begin{enumerate} \itemsep1pt \parskip0pt \parsep0pt
  \item Construct implication graph and contract it (it will be a skew-symmetric DAG)
  \item Consider the vertices in topsort order. If it was not assigned a value, set it's value to false. This causes all of the terms in the complementary component to be set to true.
  \end{enumerate}
  Due to the topological ordering, when a term x is set to false, all terms that lead to it via a chain of implications will themselves already have been set to false. Symmetrically, when a term is set to true, all terms that can be reached from it via a chain of implications will already have been set to true. Therefore, the truth assignment constructed by this procedure satisfies the given formula.
}

\SECTION {Stoer-Wagner algorithm for global mincut} {
  \Csourcefile{hashed/stoerWagner}
}

\SECTION {Maximum matching in generic graph} {
  Edmonds' algorithm: $O(n^3)$
  \Csourcefile{hashed/edmondsMatching}
}

\SECTION {Tarjan's algorithm for Strongly Connected Components} {
  \Csourcefile{hashed/tarjan}
}

\SECTION {Articulation point and Bridges} {
  \Csourcefile{hashed/bridgeAp}
}

%%%%%%%%%%%%%%%%%%%%%%%%%%%%%%%%%%%%%%%%%%%%%%%%%%%%%%%%%%%%%%%%%%%%%%%%%%%%%%%%%%%%%%
%%%%%%%%%%%%%%%%%%%%%%%%%%%%%%%%%%%%%%%%%%%%%%%%%%%%%%%%%%%%%%%%%%%%%%%%%%%%%%%%%%%%%%

\section{Strings}

\SECTION {Aho-Corasick} {
  \Csourcefile{hashed/ahoCorasick}
}

\SECTION {Manacher's algorithm for finding longest palindromic substring} {
  \Csourcefile{hashed/manacher}
}

\SECTION {Suffix array} {
  \Csourcefile{hashed/suffixArray}
}

\vspace{0.5cm}

\SECTION {Knuth-Morris-Pratt algorithm} {
  \Csourcefile{hashed/KMP}
}

\SECTION {Hash} {
  \Csourcefile{hashed/hash}
}

\SECTION {Z-algorithm} {
  \Csourcefile{hashed/z}
}

%%%%%%%%%%%%%%%%%%%%%%%%%%%%%%%%%%%%%%%%%%%%%%%%%%%%%%%%%%%%%%%%%%%%%%%%%%%%%%%%%%%%%%
%%%%%%%%%%%%%%%%%%%%%%%%%%%%%%%%%%%%%%%%%%%%%%%%%%%%%%%%%%%%%%%%%%%%%%%%%%%%%%%%%%%%%%

\section{Data structures}

\SECTION {Static Range Minimum Query} {
  \Csourcefile{hashed/rmq}
}

\SECTION {2D Longest Increasing Subsequence} {
  \Csourcefile{hashed/lis2d}
}

\SECTION {Treap} {
  \Csourcefile{hashed/treap}
}

\FINALSECTION {KD-Tree} {
  \Csourcefile{hashed/kdtree}
}
%%%%%%%%%%%%%%%%%%%%%%%%%%%%%%%%%%%%%%%%%%%%%%%%%%%%%%%%%%%%%%%%%%%%%%%%%%%%%%%%%%%%%%
%%%%%%%%%%%%%%%%%%%%%%%%%%%%%%%%%%%%%%%%%%%%%%%%%%%%%%%%%%%%%%%%%%%%%%%%%%%%%%%%%%%%%%


\end{document}

